% First parameter can be changed eg to "Glossary" or something.
% Second parameter is the max length of bold terms.
%To keep this glossary useful, we mainly focus on abbreviations specific to our domain and don't list very common abbrevations (such as NLP - Natural Language Processing).
\begin{mclistof}{Glossary}{3.2cm}
%\item[NLP] Natural Language Processing
%\item[CV] Computer Vision
\item[Splicing] Process by which a pre-mRNA is converted to a mature mRNA or the actual act of cutting out genomic sequences in that process itself.
\item[Exon] Genomic sequence which is typically kept from the pre-mRNA during splicing.
\item[Intron] Genomic sequence which is typically removed from the pre-mRNA during splicing.
\item[PSI] Percent-spliced in, in what proportions of transcript a specific exon (or junction) is contained in the mature mRNA.
\item[Motif] A widespread nucleotide sequence pattern conjectured to be biologically significant.
%\item[MLP] Multi-layer perceptron
%\item[w2v] word2vec algorithm
%\item[D2V] Either the algorithm document2vector, or the specific model based on this algorithm.
\item[EST] Expressed sequence tag, a short cDNA sequence used in older sequencing techniques.
\item[GTEx] Genotype-Tissue Expression project, they provide a large repository of sequencing data which we use.

\item[iPSC] induced pluripotent stem cells, mature cells which have been reprogrammed to an immature pluripotent (undetermined) state.
\item[HipSci] Human Induced Pluripotent Stem Cell Initiative, a repository of iPSC-based data which we use.

%\item[ROC] Receiver Operating Curve
%\item[AUC] Area under ROC curve
%\item[Seq2seq] Sequence-to-sequence learning, a framework often used in MT.
%\item[MT] Machine Translation
%\item[RNN] Recurrent Neural Network
%\item[CNN] Convolutional Neural Network
\item[DSC] Deep Splicing Code, a model used for constitutive exon classification.



\end{mclistof} 
