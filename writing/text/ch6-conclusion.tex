\chapter{\label{ch:6-conclusion}Discussion / Conclusion}


- across all datasets, the newly introduced splicing code RASC generally performs best.

say that data processing was actually the most time consuming task due to the novelty of the task and no standard datasets available

- proper dataset construction very challenging with varying quality of available tools and datasets; only one of the four datasets proved sufficient


future work: differential splicing, link other datasets
future work could also be using gtex data directly perhaps? 
differential splicing for sure
 
future work: data augmentation as performance dropped 2-3\% once i removed 10%
easy extension would be check whether more context than 140 nucleotides help
continuous splicing prediction; more than 140 nt probably desirable 


We make our dataset publicly available; related to talking about publishing this research in general 